\documentclass[10pt,a4paper]{article}

\usepackage[T1]{fontenc}
\usepackage{palatino}
\usepackage{inconsolata}

\usepackage{fullpage}

\usepackage{amsmath}
\usepackage{amssymb}
\usepackage{listings}
\lstset{language=Haskell,frame=single,basicstyle=\ttfamily,numbers=left}

\newcommand{\just}[2]{\texttt{#1} & \textrm{#2}}
\newcommand{\cycle}[1]{cycle(\texttt{#1})}
\newcommand{\reach}[2]{reach(\texttt{#1},\texttt{#2})}
\newcommand{\existstin}[2]{\exists\texttt{#1}\in\texttt{#2}}

\begin{document}

\title{PMT Additional Exercises (Week 5)}
\author{Nicholas Sim}
\date{February 8th, 2018}
\maketitle{}


\section{Maximum sum}

\begin{lstlisting}
maxSum :: [Integer] -> Integer
maxSum = snd . foldr f (0,0)
  where f x (c,g) = (c',g')
    where c' = max 0 (c+x)
          g' = max g c'
\end{lstlisting}

Prove that \texttt{maxSum} (Kadane) computes the maximum sum of a \emph{contiguous} subarray.
That is:

\[ \forall \texttt{xs:[Int].maxSum xs} = \max_{0\le i \le j < \texttt{\#xs}} \sum_{k=i}^{j} \texttt{xs}[k] \]

Remark: this algorithm is usually run left-to-right (i.e.~\texttt{foldl}), but produces the same result.


\section{Cycle-finding}
\emph{Remark: representing (potentially) cyclic data structures in Haskell can be a hazard. Search `tying the knot' for more information.}

\begin{lstlisting}
data Node a = Node { value :: a, targets :: [Node a] }

findCycle :: [Node a] -> Node a -> Bool
findCycle visited node
= elem node visited || any (findCycle node:visited) (targets x)
\end{lstlisting}

A \texttt{Node} represents a node in a connected \emph{directed} graph.
Use \(cycle(\texttt{n})\) for a predicate indicating a cycle can be reached from a node \texttt{n}, and \( \reach n m \) for \texttt{n} is reachable from \texttt{m} (both nodes).

\begin{enumerate}
	\item Write down a structural induction principle for the \texttt{Node} data type.
	\item (*) Write down a property (provable with induction) sufficient to show that \texttt{findCycle} determines if there is a cycle from a given node.
	\item (***) Complete the proof\footnote{Unlike last week, I have a written solution. You should consider this question to be at least as difficult as any exam question for Reasoning.}
\end{enumerate}


\newpage


\section{Solution to maximum sum}
This is a classic problem in computer science, and as noted,
we have implemented Kadane's algorithm, which has linear time complexity.

The first thing we should observe is that \texttt{c} represents the current sum, and \texttt{g} the global.
Similar to last week, we will instead prove a property about \texttt{f}:

\[ \forall \texttt{xs:[Int].foldr f (0,0) xs} = (\max_{0\le j < \texttt{\#xs}} \sum_{k=0}^{j} \texttt{xs}[k],\max_{0\le i \le j < \texttt{\#xs}} \sum_{k=i}^{j} \texttt{xs}[k]) \]

Recall:

\begin{lstlisting}
foldr f z []     = z
foldr f z (x:xs) = f x (foldr f z xs)
\end{lstlisting}

\paragraph{Base case}
To show: \texttt{foldr f (0,0) [] = (0,0)}

This follows directly from the definition of \texttt{foldr}.

\paragraph{Inductive step}
Fix \texttt{xs:[Int]}.

Inductive Hypothesis:

\[ \texttt{foldr f (0,0) xs} =
(\max_{0\le j < \texttt{\#xs}} \sum_{k=0}^{j} \texttt{xs}[k],
\max_{0\le i \le j < \texttt{\#xs}} \sum_{k=i}^{j} \texttt{xs}[k]) \]

We will call this tuple \texttt{(c,g)}.

To show:
\[ \forall \texttt{x:Int.foldr f (0,0) x:xs} =
(\max_{0\le j \le \texttt{\#xs}} \sum_{k=0}^{j} \texttt{(x:xs)}[k],
\max_{0\le i \le j \le \texttt{\#xs}} \sum_{k=i}^{j} \texttt{(x:xs)}[k]) \]

Claim (*) that \texttt{(c,g)} represent the variables mentioned above at each step of the fold. Alternate I.H.:

\begin{align*}
& \texttt{maxSum x:xs} \\
&= \just{snd . foldr f (0,0) x:xs}{by def. maxSum} \\
&= \just{snd . f x (foldr f (0,0) xs)}{by def. foldr} \\
&= \just{snd . f x (c,g)}{by I.H.} \\
&= \just{snd . (max 0 c+x,max g (max 0 c+x))}{by def. f} \\
&= \just{max g (max 0 c+x)}{by def. snd}
\end{align*}

I claim that this is what we want.
Cases: either \texttt{x} is in the sum, or not.
If it is, the sublist starts at \texttt{x}, thus \texttt{c+x} is the maximum.
Otherwise \texttt{g} is the maximum (by problem definition and given by I.H.).

\section{Solution to cycle-finding}
Remark: this is for reachable nodes. We only have one case (refreshing!)\footnote{Below, we use \( \forall\texttt{x}\in\texttt{xs.} P(\texttt{x}) \) as a shorthand for \( \forall\texttt{x:a.} \texttt{elem x xs} \rightarrow P(\texttt{x}) \), where \texttt{xs:[a]} and \(P\) some predicate.}

\[ \forall\texttt{n:Node.} (\forall\texttt{m}\in \texttt{targets n.} P(\texttt{m})) \rightarrow P(\texttt{n}) \]

Want to show: \( \forall\texttt{n:Node,vs:[Node].}\texttt{findCycle vs n} \leftrightarrow cycle(\texttt{n}) \lor \exists\texttt{v}\in\texttt{vs}: reach(\texttt{v},\texttt{n}) \)\footnote{Likewise, \( \exists\texttt{x}\in\texttt{xs.} P(\texttt{x}) \) represents \emph{something sensible that I am too lazy to write}. (sufficient to apply double negative to the universally quantified case to note that this is valid)}

This proof is a sketch only, write the detail in yourself.

\paragraph{Inductive step}
Fix \texttt{ts:[Node]}.

Inductive Hypothesis:
\[ \forall\texttt{m}\in\texttt{ts,vs:[Node].}\texttt{findCycle vs m} \leftrightarrow cycle(\texttt{m}) \lor \exists\texttt{v}\in\texttt{vs}: reach(\texttt{v},\texttt{m}) \]

To show:
\[ \forall \texttt{x:a,vs:[Node].}\texttt{findCycle vs n} \leftrightarrow \cycle n \lor \exists\texttt{v}\in\texttt{vs}: \reach v n \]

writing \texttt{n} for \texttt{Node \{value=x, targets=ts\}}.

Fix \texttt{x:a}, \texttt{vs:[Node]}.

%Assume none of the targets in \texttt{ts} have a reachable cycle
%(otherwise so does our node, and follows from the \texttt{any} clause in \texttt{findCycle}).

Assume LHS. Want to show \(cycle(\texttt{n}) \lor \exists\texttt{v}\in\texttt{vs}: reach(\texttt{v},\texttt{n})\).
Assume \( \texttt{n}\notin\texttt{vs} \), otherwise \(reach\) immediately follows from \texttt{elem node visited}.
So \( \exists\texttt{m}\in\texttt{ts}: \texttt{findCycle n:vs m} \).
Applying I.H. we get \( \cycle m \) (so \(\cycle n \)) or \(\exists\texttt{v}\in\texttt{(n:vs)}: \reach v m \) (in which case \(\exists\texttt{v}\in\texttt{vs}: \reach v n \), or \(\reach n n \), i.e. \(\cycle n\)).

Assume \( \existstin v {vs}: \reach v n \).
Then certainly either \(\texttt{n}\in\texttt{vs}\) (hence \texttt{elem n vs}), or \(\existstin m {ts} : \existstin v {vs} : \reach v m \) (fulfil \texttt{any} case using I.H.). Thus LHS holds.

Finally, for the substance of the proof: assume \( \cycle n \).
Clearly we have \( \reach n n \) (\texttt{n} is part of a cycle), or \( \existstin m {ts} : \cycle m \).
If the latter holds, then \( \forall \texttt{vs:[Node].} \existstin m {ts} : \texttt{findCycle vs m} \) by I.H., so the \texttt{any} case holds.
Otherwise, \texttt{n} is in \texttt{ts} (trivial application of the \texttt{elem} case), or \( \existstin m {ts}: \reach n m \) (by reasoning about cycles and reachability).
But now \texttt{findCycle n:vs m} holds since \( \exists v {(n:vs)} : \reach v m \) (I.H.), so \texttt{findCycle vs n} holds by the \texttt{any} case.

\end{document}
