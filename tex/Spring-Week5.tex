\documentclass[10pt,a4paper]{article}

\usepackage[T1]{fontenc}
\usepackage{palatino}
\usepackage{inconsolata}

\usepackage{fullpage}

\usepackage{amsmath}
\usepackage{amssymb}
\usepackage{listings}
\lstset{language=Haskell,frame=single,basicstyle=\ttfamily,numbers=left}

\newcommand{\just}[2]{\texttt{#1} & \textrm{#2}}

\begin{document}

\title{PMT Additional Exercises (Week 5)}
\author{Nicholas Sim}
\date{February 8th, 2018}
\maketitle{}


\section{Maximum sum}

\begin{lstlisting}
maxSum :: [Integer] -> Integer
maxSum = snd . foldr f (0,0)
  where f x (c,g) = (c',g')
    where c' = max 0 (c+x)
          g' = max g c'
\end{lstlisting}

Prove that \texttt{maxSum} (Kadane) computes the maximum sum of a \emph{contiguous} subarray.
That is:

\[ \forall \texttt{xs:[Int].maxSum xs} = \max_{0\le i \le j < \texttt{\#xs}} \sum_{k=i}^{j} \texttt{xs}[k] \]

Remark: this algorithm is usually run left-to-right (i.e.~\texttt{foldl}), but produces the same result.


\newpage

\section{Solution to trees}

\section{Solution to maximum sum}
This is a classic problem in computer science, and as noted,
we have implemented Kadane's algorithm, which has linear time complexity.

The first thing we should observe is that \texttt{c} represents the current sum, and \texttt{g} the global.
Similar to last week, we will instead prove a property about \texttt{f}:

\[ \forall \texttt{xs:[Int].foldr f (0,0) xs} = (\max_{0\le j < \texttt{\#xs}} \sum_{k=0}^{j} \texttt{xs}[k],\max_{0\le i \le j < \texttt{\#xs}} \sum_{k=i}^{j} \texttt{xs}[k]) \]

Recall:

\begin{lstlisting}
foldr f z []     = z
foldr f z (x:xs) = f x (foldr f z xs)
\end{lstlisting}

\paragraph{Base case}
To show: \texttt{foldr f (0,0) [] = (0,0)}

This follows directly from the definition of \texttt{foldr}.

\paragraph{Inductive step}
Fix \texttt{xs:[Int]}.

Inductive Hypothesis:

\[ \texttt{foldr f (0,0) xs} =
(\max_{0\le j < \texttt{\#xs}} \sum_{k=0}^{j} \texttt{xs}[k],
\max_{0\le i \le j < \texttt{\#xs}} \sum_{k=i}^{j} \texttt{xs}[k]) \]

We will call this tuple \texttt{(c,g)}.

To show:
\[ \forall \texttt{x:Int.foldr f (0,0) x:xs} =
(\max_{0\le j \le \texttt{\#xs}} \sum_{k=0}^{j} \texttt{(x:xs)}[k],
\max_{0\le i \le j \le \texttt{\#xs}} \sum_{k=i}^{j} \texttt{(x:xs)}[k]) \]

Claim (*) that \texttt{(c,g)} represent the variables mentioned above at each step of the fold. Alternate I.H.:

\begin{align*}
& \texttt{maxSum x:xs} \\
&= \just{snd . foldr f (0,0) x:xs}{by def. maxSum} \\
&= \just{snd . f x (foldr f (0,0) xs)}{by def. foldr} \\
&= \just{snd . f x (c,g)}{by I.H.} \\
&= \just{snd . (max 0 c+x,max g (max 0 c+x))}{by def. f} \\
&= \just{max g (max 0 c+x)}{by def. snd}
\end{align*}

I claim that this is what we want.
Cases: either \texttt{x} is in the sum, or not.
If it is, the sublist starts at \texttt{x}, thus \texttt{c+x} is the maximum.
Otherwise \texttt{g} is the maximum (by problem definition and given by I.H.).

\end{document}
