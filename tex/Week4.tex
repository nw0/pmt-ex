\documentclass[10pt,a4paper]{article}
    
\usepackage{fullpage}

\usepackage{amsmath}
\usepackage{amssymb}

\begin{document}

\title{PMT Additional Exercises (Week 4)}
\author{Nicholas Sim}
\date{October 26th, 2017}
\maketitle{}


\section{Discrete Structures: Equivalence Classes}
% Relations (repeat)
% Equivalence classes

In the lectures you have constructed the rational numbers 
\( \mathbb{Q} \) from the sets \( \mathbb{Z}, \mathbb{N} \).
Here we construct \( \mathbb{Z} \), since we only understand 
\( \mathbb{N} \). (This is glossed over in def. 3.4.)

Let \( S = \mathbb{N} \times \mathbb{N} \). 
Define a relation \( \sim \) on \( S \) by \( (m_1, n_1) \sim (m_2, n_2) \) 
(or \( \left( (m_1, n_1), (m_2, n_2) \right) \in \sim \))
iff \( m_1 + n_2 = m_2 + n_1 \).
Informally, we can think of \( (m, n) \) as \( m - n \).

\emph{[construction thanks to A. Corti]}

\begin{enumerate}
    \item Show that \( \sim \) is an equivalence relation.
    \item Show that \( (m_1, n_1) \sim (m_2, n_2) \) iff 
    \( (m_1 + k, n_1 + k) \sim (m_2 + k, n_2 + k) \) 
    for any \( k \in \mathbb{N} \).
    \item Show that if \( (m_1, k) \sim (m_2, k) \) then 
    \( m_1 = m_2 \) 
    for any \( k \in \mathbb{N} \).
    \item Similarly, show that if \( (k, n_1) \sim (k, n_2) \) then 
    \( n_1 = n_2 \) 
    for any \( k \in \mathbb{N} \).

    Now let \( Z = S / \sim \), and define \( + \) on \( Z \) by 
    \( [(m_1, n_1)] + [(m_2, n_2)] = [(m_1 + m_2, n_1 + n_2)] \).
    
    For convenience, represent members of \( Z \) by 
    \( \bar{z} = [(z, 0)] \) and \( -\bar{z} = [(0, z)] \)
    for any \( z \in \mathbb{N} \).

    \item Show that \( \bar{z} + (-\bar{z}) = [(0, 0)] \).
    \item Fix any \( \bar{z} \in Z \). 
    Show that either \( z \in \mathbb{N} \) or (abusing notation) 
    \( -z \in \mathbb{N} \).
    \item (*) Write down an invertible function 
    \( f : \mathbb{Z} \rightarrow Z \).
    Show that \( \forall a, b \in \mathbb{Z}, f (a + b) = f(a) + f(b) \).
\end{enumerate}


\section{Logic: Adequacy}
% Lectures: Equivalences
% This is quite well-covered by the tutorial sheets
% We focus on adequacy and some proofs, e.g. EM, contrapositive


[Informal definition]
Say a set of connectives is \emph{adequate} if any propositional formula 
of \(n\) variables can be written as some other formula only using variables 
\(p_1, ... , p_n\) and the connectives.

For instance, \( \{ \neg , \lor \}, \{ \neg, \rightarrow \} \) 
are both adequate.
We've also seen that \( \{ \uparrow \} \) (NAND) is adequate.

\begin{enumerate}
    \item We introduce a new connective, NOR (\( \downarrow \)).
    Show that this connective is adequate.
    \item (*) Show that apart from NOR and NAND, 
    there are no other single adequate (binary) connectives.
    \emph{[thanks to D. Evans]}
\end{enumerate}


\newpage

\section{Solutions to Equivalence Classes}

\begin{enumerate}
    \item
    Reflexivity. Clearly \( m_1 + n_1 = m_1 + n_1 \). \\
    Symmetry. Suppose \( m_1 + n_2 = m_2 + n_1 \).
    Then clearly \( m_2 + n_1 = m_1 + n_2 \). \\
    Transitivity. Suppose \( m_1 + n_2 = m_2 + n_1 \) and 
    \( m_2 + n_3 = m_3 + n_2 \). Then
    \( m_1 + n_3 = (m_2 + n_1 - n_2) + (m_3 + n_2 - m_2) = n_1 + m_3 \).
    \item We have \( m_1 + n_2 = m_2 + n_1 \).
    For \( \Rightarrow \), verify by adding \(k\) to all four terms.
    For \( \Leftarrow \), fix \( k = 0 \).
    \item Suppose \( m_1 + k = m_2 + k \). Subtract \(k\) from both sides 
    (valid as each side \( \ge k \)).
    \item Similar to previous part.
    \item This is just \( [(z, 0)] + [(0, z)] = [(z + 0, 0 + z)] = [(z, z)]\) 
    as defined by \( + \) on \( Z \). Trivially we have 
    \( (k, k) \sim (0, 0) \forall k \in \mathbb{N} \) since \( k + 0 = 0 + k\).
    So clearly \( [(z, z)] = [(0, 0)]) \) since they belong to the same 
    equivalence class.
    \item Write \( \bar{z} = [(m, n)] \). Either \( m \ge n \), then 
    \( \bar z = [(m, n)] = [(m - n, 0)] \), or \( m \le n \), where 
    \( \bar z = [(m, n)] = [(0, n - m)] \). Note that in the case \( m = n \), 
    both are valid and of course \( \bar z \sim (0, 0) \).
    \item Define \(f(x) = [(x, 0)] \) when \(x \ge 0 \) and 
    \(f(x) = [(0, -x)] \) otherwise. Simply enumerate 4 possibilities:
    \begin{enumerate}
        \item \( a, b \ge 0 \).
        \( f (a + b) = [(a + b, 0)] = [(a, 0)] + [(b, 0)] = f(a) + f(b) \).
        \item \( a < 0 \le b \).
        \( f (a + b) = f ( b - (-a) ) = [(b, -a)] = [(b, 0)] + [(0, -a)] = f(a) + f(b) \).
        \item \( b < 0 \le a \). Reverse \(a, b\) above.
        \item \( a, b < 0 \).
        \( f (a + b) = f ( 0 - ( - (a + b))) = [(0, -(a + b)] = [(0, -a)] + [(0, -b)] = f(a) = f(b) \).
    \end{enumerate}
    Slightly tedious.
\end{enumerate}

\section{Solutions to Adequacy}

\begin{enumerate}
    \item We need only use NOR to replicate a set of adequate connectives.
    Write down: \( \neg p \equiv p \downarrow p \) and 
    \( p \land q \equiv ( p \downarrow p ) \downarrow ( q \downarrow q) \).
    \item Note: there are \(2^4\) possible binary connectives.
    Suppose that a binary connective \( \cdot \) is adequate. We know that 
    \( \top \cdot \top \equiv \bot \) and \( \bot \cdot \bot \equiv \top \), 
    otherwise we would be unable to express negation.
    Only 4 possibilities remain, of which two are NOR and NAND.
    In the remaining two cases, \( p \cdot q \) would be logically equivalent 
    to either \( \neg p \) or \( \neg q \) (draw out the truth table!), 
    which isn't adequate by itself.
\end{enumerate}

\end{document}