\documentclass[10pt,a4paper]{article}

\usepackage[T1]{fontenc}
\usepackage{palatino}
\usepackage{inconsolata}

\usepackage{fullpage}

\usepackage{amsmath}
\usepackage{amssymb}
\usepackage{listings}
\lstset{language=Haskell,frame=single,basicstyle=\ttfamily,numbers=left}

\newcommand{\just}[2]{\texttt{#1} & \textrm{#2}}

\begin{document}

\title{PMT Additional Exercises (Week 4)}
\author{Nicholas Sim}
\date{February 1st, 2018}
\maketitle{}

% This week's assessed PMT is on induction on lists and graphs
% They haven't done induction on graphs in the lectures yet (?)
% But I think the Discrete Structures course on 1st term provides enough foundation

\section{Tree Reversal}
[2010 Q1b (C146)] Consider the following operations on trees and lists:

\begin{lstlisting}
data Tree a = Leaf a | Node (Tree a) (Tree a)

flatten :: Tree a -> [a]
flatten Leaf x = [x]
flatten (Node t1 t2) = (flatten t1) ++ (flatten t2)

rotate :: Tree a -> Tree a
rotate Leaf x = Leaf x
rotate (Node t1 t2) = Node (rotate t2) (rotate t1)

rev :: [a] -> [a]
rev [] = []
rev (x:xs) = (rev xs) ++ [x]
\end{lstlisting}

Prove, \emph{using structural induction}, that:

\[ \forall \texttt{t:Tree a. flatten (rotate t) = rev (flatten t)} \]

In the proofs, state what is given, the induction hypothesis (if any), 
what is to be shown, and justify each step.
You may use the property \((P)\), where:

\[ \texttt{rev (xs ++ ys) = (rev ys) ++ (rev xs)} \]


\section{Stronger properties}
[RAP 2018 PMT tutorial W4 Q2, adapted]

\begin{lstlisting}
revB :: [a] -> [a] -> [a]
revB ys [] = ys
revB ys (x:xs) = revB (x:ys) xs
\end{lstlisting}

We want to prove that \texttt{(revB [])} reverses a list, i.e.:

\[ \forall \texttt{xs:[a]. revB [] xs = rev xs} \]

However, this fails in the inductive step using structural induction over \texttt{xs}. Why?
What alternative lemma can we prove from which we can derive this?

Bonus question: complete the proof.


\newpage

\section{Solution to trees}
A straightforward structural induction (to prove: as in question) on trees.
Note that the type \texttt{a} does not matter in this question; consider it fixed.

\paragraph{Base case}
To show: \texttt{flatten (rotate (Leaf x)) = rev (flatten (Leaf x))}

\begin{align*}
& \texttt{flatten (rotate (Leaf x))} \\
&= \just{flatten (Leaf x)}{by def. rotate} \\
&= \just{[x]}{by def. flatten} \\
&= \just{rev [x]}{by def. rev} \\
&= \just{rev (flatten (Leaf x))}{by def. flatten}
\end{align*}

\paragraph{Inductive step}
Fix \texttt{t = Node t1 t2}.
Inductive Hypothesis:

\[ \forall \texttt{flatten (rotate t1) = rev (flatten t1)} \]
\[ \land \forall \texttt{flatten (rotate t2) = rev (flatten t2)} \]

To show:
\[ \texttt{flatten (rotate (Node t1 t2)) = rev (flatten (Node t1 t2))} \]

\begin{align*}
& \texttt{flatten (rotate (Node t1 t2))} \\
&= \just{flatten (Node (rotate t2) (rotate t1))}{by def. rotate} \\
&= \just{(flatten (rotate t2)) ++ (flatten (rotate t1))}{by def. flatten} \\
&= \just{(rev (flatten t2)) ++ (rev (flatten t1))}{by I.H} \\
&= \just{rev (flatten t1) ++ (flatten t2)}{by (P)} \\
&= \just{rev (flatten (Node t1 t2))}{by def. flatten}
\end{align*}


\section{Solution to stronger properties}
With \texttt{xs} fixed, our would-be inductive hypothesis would look like this:

\[ \texttt{revB [] xs = rev xs} \]

To show:

\[ \forall \texttt{x:Integer. revB [] (x:xs) = rev (x:xs)} \]

\begin{align*}
& \texttt{revB [] (x:xs)} \\
&= \just{revB (x:[]) xs}{by def. revB} \\
&= ???
\end{align*}

Prove this instead (structural induction over \texttt{xs} again):

\[ \forall \texttt{xs:[Integer],ys:[Integer]. revB ys xs = (rev xs) ++ ys} \]

\end{document}
