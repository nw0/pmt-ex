\documentclass[10pt,a4paper]{article}
    
\usepackage{fullpage}

\usepackage{amsmath}
\usepackage{amssymb}

\begin{document}

\title{PMT Additional Exercises (Week 9)}
\author{Nicholas Sim}
\date{November 30th, 2017}
\maketitle{}
% Last week we talked about equality in first-order languages
%  and groups as first-order structures.

% This week's programme:
% - Induction (by request)
% - Some revision for DS (adapted past exam qns)
% Consider doing more DS next week, rather than logic.

\section{Induction}

\section{Discrete Structures: Selected questions}
All questions are verbatim or adapted; sources indicated.

\begin{enumerate}
    \item \emph{[2008 exam Q3d, 9 mins]}
    Say that a binary relation \(\sim \subset A \times A \)  is \emph{combinative} if it fulfils:
    \[ (\forall x, y, z) x \sim z \land y \sim z \Rightarrow y \sim x \]
    Prove that if \(R\) is reflexive and combinative, then \(R\) is an equivalence relation.

    \item \emph{[2007 exam Q2c, 6 mins]}
    Given that \(A, B\) are transitive relations, show that \( A \cap B \) is also transitive.

    \item \emph{[2015 Q1e(iii) and 2016 Q1c(iii), 3 mins]}
    Let \(f: A \rightarrow B, g : B \rightarrow C\) be (total) functions.
    Prove that if \(g \circ f\) is surjective, then so is \(g\).

    \item \emph{[General, 4 mins]}
    For each of the following functions, write down whether they are injective or surjective individually.
    \begin{enumerate}
        \item \( f : \mathbb{R} \rightarrow \mathbb{R}^+, f(x) = e^x \)
        \item \( f : [0, 0.5] \rightarrow [0, 1], f(x) = x - x^2 \)
    \end{enumerate}
\end{enumerate}

\newpage

\section{Solutions to induction}

\section{Solutions to revision}
\begin{enumerate}
    \item To show: \(R\) is an equivalence relation. We will call the combinative property (C).\\
    (R) Reflexivity: given. \\
    (S) Symmetry: Suppose \( a R b \). We have \(b R b\) from reflexivity.
    Substituting \(b\) for \(z\), \(a\) for \(x\) in (C), we have \(b R a\). \\
    (T) Transitivity: Suppose \( a R b, b R c \). We want to show \( a R c \).
    We have \(c R b\) by (S) on \(b R c\), and \(a R b\) (given).
    Substituting \(c, a, b\) for \(x, y, z\) respectively in (C), we get \(a R c\), as required.
\end{enumerate}

\end{document}