\documentclass[10pt,a4paper]{article}
    
\usepackage{fullpage}

\usepackage{amsmath}
\usepackage{amssymb}

\begin{document}

\title{PMT Additional Exercises (Week 9)}
\author{Nicholas Sim}
\date{November 30th, 2017}
\maketitle{}
% Last week we talked about equality in first-order languages
%  and groups as first-order structures.

% This week's programme:
% - Induction (by request)
% - Some revision for DS (adapted past exam qns)
% Consider doing more DS next week, rather than logic.

\section{Induction}
Prove (by induction) that every positive integer greater than 1 is a product of primes.

\section{Discrete Structures: Selected questions}
All questions are verbatim or adapted; sources indicated.

\begin{enumerate}
    \item \emph{[2008 exam Q3d, 9 mins]}
    Say that a binary relation \(\sim \subset A \times A \)  is \emph{combinative} if it fulfils:
    \[ (\forall x, y, z) x \sim z \land y \sim z \Rightarrow y \sim x \]
    Prove that if \(R\) is reflexive and combinative, then \(R\) is an equivalence relation.

    \item \emph{[2007 exam Q2c, 6 mins]}
    Given that \(R, S\) are transitive relations on a domain \(A\), show that \( R \cap S \) is also transitive.

    \item \emph{[2015 Q1e(iii) and 2016 Q1c(iii), 3 mins]}
    Let \(f: A \rightarrow B, g : B \rightarrow C\) be (total) functions.
    Prove that if \(g \circ f\) is surjective, then so is \(g\).

    \item \emph{[General, 4 mins]}
    For each of the following functions, write down whether they are injective or surjective individually.
    \begin{enumerate}
        \item \( f : \mathbb{R} \rightarrow \mathbb{R}^+, f(x) = e^x \)
        \item \( f : [-1, 0.5) \rightarrow [-2, 0.25], f(x) = x - x^2 \)
    \end{enumerate}

    \item \emph{[Selected M1F (A. Corti), no more than 1 min each]}
    True or false:
    \begin{enumerate}
        \item Every subset \(S \subset \mathbb{N} \) has a least element (under the ordering \( \le \)).
        \item Let \(f : A \rightarrow B \) be a function.
        \( R = \left\{ (a_1, a_2) \in A \times A : f(a_1) = f(a_2) \right\} \) 
        is an equivalence relation on \(A\).
        \item Let \( f : A \rightarrow B \) be a function.
        Suppose that there is a function \( g : B \rightarrow A \) such that 
        \( (\forall a \in A) g \circ f (a) = a \), and another function \( h : B \rightarrow A \) with 
        \( (\forall b \in B) f \circ h (b) = b \). Then \(g = h\).
        \item Let \(X\) be a finite set. The set of all functions \( f : \mathbb{N} \rightarrow X \) is countable.
        \item Let \(X\) be a finite set. The set of all functions \( f : X \rightarrow \mathbb{N} \) is countable.
    \end{enumerate}
\end{enumerate}

\newpage

\section{Solutions to induction}

\section{Solutions to revision}
Note: there is a distinct possibility that any of the solutions may be incorrect or insufficient.

\begin{enumerate}
    \item To show: \(R\) is an equivalence relation. We will call the combinative property (C).\\
    (R) Reflexivity: given. \\
    (S) Symmetry: Suppose \( a R b \). We have \(b R b\) from reflexivity.
    Substituting \(b\) for \(z\), \(a\) for \(x\) in (C), we have \(b R a\). \\
    (T) Transitivity: Suppose \( a R b, b R c \). We want to show \( a R c \).
    We have \(c R b\) by (S) on \(b R c\), and \(a R b\) (given).
    Substituting \(c, a, b\) for \(x, y, z\) respectively in (C), we get \(a R c\), as required.

    \item \(R, S\) both subsets of \(A \times A\).
    Transitivity means that \( (\forall a, b, c \in A) \left( (a,b), (b,c) \in R \Rightarrow (a, c) \in R \right) \),
    and similarly for \(S\).
    Now suppose \((a,b), (b,c) \in R \cap S\). Then \((a,b), (b,c) \in R\) and \((a,b), (b,c) \in S\).
    By transitivity, \((a, c) \in R\) and \((a, c) \in S\), and by definition of intersection, 
    \((a, c) \in R \cap S\). So \(R \cap S\) is also transitive.

    \item Suppose \( g \circ f \) is surjective.
    Then \( (\forall c \in C) (\exists a \in A) g(f(a)) = c \) (*). \\
    To show: \(g \) is surjective, i.e. \( (\forall c \in C) (\exists b \in B) g(b) = c \).
    But this is clear: we can set \( b = f(a) \), as existence is guaranteed by (*).
    
    \item 
    \begin{enumerate}
        \item \( f : \mathbb{R} \rightarrow \mathbb{R}^+, f(x) = e^x \). Injective and surjective.
        \item \( f : [-1, 0.5) \rightarrow [-2, 0.25], f(x) = x - x^2 \).
        Injective as monotonic increasing. Not surjective as 0.25 not mapped.
    \end{enumerate}

    \item Informal (1 minute) `proofs'
    \begin{enumerate}
        \item False. \( \varnothing \).
        \item True. Check R, S, T.
        \item True.
        Notice that \( |B| \ge |A| \) as \( g \circ f \) is the identity map.
        Likewise \( |A| \ge |B| \).
        Now satisfy yourself that \( g = f^{-1} = h \).
        \item False. Let \(X = \{ 0, 1 \} \).
        There is a bijection between the set of functions \( f : \mathbb{N} \rightarrow X \) 
        and the power set of \(\mathbb{N}\).
        \item True. Let \(|X| = n\). Easy proof by induction: \\
        Base case: \(n=1\) trivial (as is \(n=0\)) \\
        Inductive step: Assume countably many \( f : \{0, 1, \cdots , k\} \rightarrow \mathbb{N} \), 
        ordered \( f_0, f_1, \cdots \). Now there are countably many choices for \( f_i (k+1) \).
        The Cartesian product of two countable sets is also countable, and the result follows.
    \end{enumerate}
\end{enumerate}

\end{document}