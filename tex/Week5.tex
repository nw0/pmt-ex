\documentclass[10pt,a4paper]{article}
    
\usepackage{fullpage}

\usepackage{amsmath}
\usepackage{amssymb}

\begin{document}

\title{PMT Additional Exercises (Week 5)}
\author{Nicholas Sim}
\date{November 2nd, 2017}
\maketitle{}


\section{Discrete Structures: Functions}
% Peano Arithmetic and Functions
% I want to set a simple induction question, but can't think of one.
% Therefore we're going to mix equivalence classes and functions.

We have seen in the lectures that if \( R \) is an equivalence relation on 
a set \( A \), it will partition \( A \) into its quotient set \( A / R \).
Prop. 2.17 shows that \( A / R = \left\{ [a] : a \in A \right\} \).

Suppose that every equivalence class \( [a] \) has a 
(unique, distinguished) representative \( \bar{a} \).

\begin{enumerate}
    \item Let \(A, B\) be finite sets, and define 
    \(f : A \rightarrow B \) to be 
    surjective.\footnote{\emph{surjective} here means every element in 
    \(B\) is mapped by at least 1 element in \(A\).}
    Show that there is an equivalence relation \(R\) on \(A\), 
    where \( (a_1, a_2) \in R \) if and only if \( f(a_1) = f(a_2) \).

    \item Let \( g : B \rightarrow A \) be a function where 
    \( g(b) = \bar a \), the unique representative of \([a]\) 
    where \(f(a) = b\). Show that \(g\) is 
    injective.\footnote{\emph{injective} maps preserve distinctness: 
    if \( g(b_1) = g(b_2) \), then we can conclude \( b_1 = b_2 \). 
    Say: ``No two \(b\)s map to the same \(a\)."}

    \item Define \( g' : B \rightarrow A/R \) similarly to \(g\), 
    except mapping to the equivalence class \([a]\) instead of \( \bar a \). 
    Show that \(g'\) is a bijection. \\

    The following questions test your understanding of the definitions.
    \item (*) Suppose \( A \ne \varnothing \), and \( f : A \rightarrow B \). 
    Show that there is a function \( g : B \rightarrow A \) such that 
    \( (g \circ f) (a) = a \) for any \(a \in A\).
    \emph{[thanks to A. Corti]}

    \item Let \(A, B\) be finite sets of \(m, n\) elements respectively.
    \begin{enumerate}
        \item (*) How many injections \( A \rightarrow B\) are there?
        \item Let \( n = 2 \) How many surjections?
        \item (**) Now \( n > 2 \). How many surjections?
        \item How many bijections?
    \end{enumerate}
    \emph{[Hint: consider cases \(m\) larger than \(n\), etc.]}
\end{enumerate}


\section{Logic: Natural Deduction}
% Natural Deduction

Prove the following using Natural Deduction:

\begin{enumerate}
    \item \( A \rightarrow B , A \rightarrow \neg B \vdash \neg A \)
    \item \( \neg B \rightarrow \neg A \vdash A \rightarrow B \) (the contrapositive)
    \item \( \vdash A \rightarrow A \)
    \item \( \vdash A \lor \neg A \) (obviously, you're not allowed to use EM for this one)
    \item \( \left( A \rightarrow (B \rightarrow C ) \right) \vdash 
    \left( (A \rightarrow B) \rightarrow (A \rightarrow C) \right) \)
    \item \emph{[Q1a of the 2015 exam]} \( B, \neg C \rightarrow \neg A \lor \neg B \vdash A \rightarrow C \)
    \item \( A \rightarrow ( B \rightarrow C ) \vdash B \rightarrow ( A \rightarrow C) \)
\end{enumerate}

\newpage

\section{Solutions to Functions}

\begin{enumerate}
    \item This is a routine verification of R, S, T, which I will omit.
    \item Since \(f\) is surjective, \(g\) is injective. \\
    (another proof) We defined \(R\) such that \( f(a_1) = f(a_2) = b \) 
    if and only if \(a_1, a_2\) belong to the same equivalence class. 
    So \(f\) maps all the members of \( [a] \) to the same value of \(b\).
    Now suppose \( g(b_1) = g(b_2) = \bar a \in A \). 
    Then \( b_1 = f(\bar a) = b_2 \).
    \item It suffices to show that \(g'\) is surjective 
    (argue injective via surjectivity of \(f\), 
    alternatively, proof similar to injectivity of \(g\)). 
    Suppose for a contradiction that \(g'\) is not surjective. 
    Then \( \exists a \in A \) such that \( f(a) \notin B \). 
    But \(f\) is a function (hence maps all the elements in its domain).

    \item We need \( A \ne \varnothing \), because to be a function, 
    \(g\) must map any value in \(B\) to some value.
    Define \(g\) as follows.  
    Fix \( b \in B \). If \( b \notin \mathrm{image} (f) \), 
    then let \( g(b) = a_0 \) (choose any \(a_0 \in A\)). 
    If \(b\) is in the image of \(f\), then 
    \( \exists a_b \in A \) s.t. \( f (a_b) = b \). 
    Let \( g(b) = a_b \).
    
    (proof) Pick \(a \in A\) arbitrarily and verify \( (g \circ f)(a) = a \).
    Let \( f(a) = b \) and clearly \(a = a_b\) by construction. 
    Then \( g ( f(a_b) ) = g(b) = a_b = a \) as required.

    \item \( |A| = m, |B| = n \).
    \begin{enumerate}
        \item Suppose \( m > n \). There can be no injections.
        For \( m \le n \), we need to choose \(m\) distinct elements to 
        map to, from \(n\) possible values, where order (permutation) matters. 
        So \( \frac{n!}{(m-n)!} \).

        \item Try inclusion-exclusion. There are \( n ^m \) possible 
        functions \( A \rightarrow B \) (\(n\) choices for \(m\) inputs). 
        But \( n = 2 \), so just \( 2 ^ m \). There are only two possible 
        non-surjective functions (send all elements in \(A\) to the first 
        value, or the second value). So \( 2 ^ m - 2 \).

        \item (Why can't we just count? Think about this) \\
        Consider \( n = 3 \). Write \(1, 2, 3\) for the elements of \(B\).
        We need to subtract the non-surjective functions from \( 3 ^ m \). 
        There are \( 2 ^ m \) functions each including only (1, 2), (1, 3), 
        or (2, 3). But the Principle of Inclusion-Exclusion tells us that 
        we have subtracted 3 functions twice. So \( 3^m - 3 (2^m) + 3 \).

        Construction: we want \( n^m - (n)(n-1)^m + {n \choose 2} (n-2)^m - \ldots \)

        Claim: This formula works\\
        \[ \sum_{k=0}^{n} (-1)^k {n \choose k} (n-k)^m \]
        Satisfy yourself that this is correct.

        \item This is just the number of injections when \( m = n \), i.e. \(m!\).
    \end{enumerate}
\end{enumerate}


\section{Solutions to Natural Deduction}

Full proofs are not included, but you can ask me for clarifications.

\begin{enumerate}
    \item Suppose \(A\). Then from the givens, we can use 
    \( \rightarrow E \) to get both \(B, \neg B\). Contradiction. 
    So we get \( \neg A \) by \( \neg I\).

    \item Suppose \(A\). We want to get \(B\), so suppose \( \neg B\). 
    Get \( \neg A \) by \( \rightarrow E \), then \(B\) by \(PC\). 
    Conclude \(A \rightarrow B\) using \( \rightarrow I \).

    \item This one can't be that hard! Do it yourself.

    \item Note: if we could use equivalences, 
    this is exactly the same as the last. 
    This appears in the notes, and the overall form is a proof 
    by contradiction.

    \item This is slightly messy, but we'll just try: \\
    Suppose \( A \rightarrow B \). We want to get \( A \rightarrow C \). 
    So suppose \(A\). \(B\) follows using \( \rightarrow E \). 
    \( B \rightarrow C \) from using \( \rightarrow E \) on the given. 
    Then we get \(C\), thus \( A \rightarrow C \).
    Finally we have \( \left( (A \rightarrow B) \rightarrow (A \rightarrow C) \right) \).

    \item Suppose \(A\). We want to get \(C\). 
    Suppose \( \neg C \). Then we get \( \neg A \lor \neg B \). 
    Perform \( \lor E \) to get \( \neg B \): 
    but this contradicts the given \(B\). 
    So after some \( \bot I \), we get \(C\) via \(PC\).
    And that's all we need to conclude \( A \rightarrow C \).

    \item Suppose \(B\). We will show \( A \rightarrow C\).
    Suppose \(A\). Then we have \( B \rightarrow C \), but we have 
    \(B\), so we have \(C\).
\end{enumerate}

\end{document}