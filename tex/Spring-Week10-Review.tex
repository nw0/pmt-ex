\documentclass[10pt,a4paper]{article}

\usepackage[T1]{fontenc}
\usepackage{palatino}
\usepackage{inconsolata}

\usepackage{fullpage}

\usepackage{amsmath}
\usepackage{amssymb}

\begin{document}

\title{PMT Revision (Week 10)}
\author{Nicholas Sim}
\date{March 15th, 2018}
\maketitle{}


\section{Logic}
[See handout: C140 2016 Q1c]


\section{Relations}
\begin{enumerate}
	\item \emph{[C142 2015 Q1d.v]} Let \(R \subseteq A^2\).
	Prove that if \(R\) is symmetric, then \(R\circ R\) is also symmetric. 
	\item \emph{[C142 2005 Q2b.ii]} Let \(R\) be a binary relation on the natural numbers defined by \(xRy\) if and only if \(y = 2x\).
	Describe the transitive closure of \(R\).
	\item \emph{[C142 2005 Q2b.iii]} A binary relation \(R\) is antisymmetric if and only if \((a,b) \in R \) and \((b, a) \in R\) implies \(a=b\).
	How many binary relations on \(A = \{1, 2, 3, 4\}\) are both symmetric and antisymmetric? Justify your answer. \\
	\emph{[Bonus: What about an arbitrary set?]}
\end{enumerate}


\section{Functions}
\emph{[C142 2002 Q1c]} Let \(f: A\rightarrow B\) and \(g: B \rightarrow C \) be functions.
Prove the following (otherwise give a counterexample):
\begin{enumerate}
	\item \(f\) surjective implies \(g \circ f\) surjective;
	\item \(g \circ f\) injective implies \(f\) injective;
	\item \(g \circ f\) injective implies \(g\) injective;
	\item \(f, g\) surjective implies \(g \circ f\) surjective.
\end{enumerate}


\newpage

Note: these solutions are only outlines.

\section{Solution to Alternative Elimination}
\begin{enumerate}
	\item I would argue that (2-4) are sufficient for \(A \rightarrow B\).
	From here, we can use EM to get \( A \lor \neg A\), and \(\lor E\) to get \(B\).
	\item First use \(\lor I\) to get \( A \lor B \lor C\). Assume \(A\) to get \(C\) then \(B \lor C\) (using \(\lor I\)). Now use \(Alt\lor E\) to get \(B \lor C\). Use \(Alt\lor E\) again to get \(C\).
	\item Observe that \(\lor E\) essentially follows from the previous part. (Say: if we have ..., then get \(\vdash^a C\))
	\item \(\vdash^a\) is complete if every valid statement expressed in propositional logic can be proven using only the rules of \(\vdash^a\).
	It is sound if every statement that can be proven under the system is valid.
	\item We know from the lectures that \(\vdash\) is sound and complete.
	Since \(Alt\lor E\) is a derived rule of \(\vdash\), soundness is preserved (we can't prove anything new, hence nothing invalid).
	Since \(\lor E\) is a derived rule of \(\vdash^a\), completeness is preserved (we can prove at least the same things that \(\vdash\) can).
\end{enumerate}


\section{Solutions to Relations}
\begin{enumerate}
	\item Suppose that \(R\) is symmetric, i.e. \( (a, b) \in R \Rightarrow (b,a) \in R \). Recall that \( R \circ R = \{ (a,c) : (\exists b) \left( (a,b) \in R \land (b,c) \in R \right) \} \). \\
	Fix \( (a,c) \in R \circ R \). To show: \( (c,a) \in R \circ R \). But this is clear as \( \exists b (a,b), (b,c) \in R \), so \( (b,a), (c,b) \in R \) by symmetry, and \( (c,a) \in R \circ R\).
	\item \(x \bar R y\) if and only if \(\exists n \in \mathbb{N}_{>0} : y = 2^n x \).
	\item \(R\) is symmetric if and only if \( (b,a) \in R \) whenever \( (a,b) \in R \). Let \(R\) be both symmetric and antisymmetric. Suppose \( (a,b) \in R \). Then \( (b,a) \in R \), so \(a=b\). \\
	We easily see that the possible relations \(R\) are those which only contain pairs of the form \( (a,a) \). So 16.
\end{enumerate}


\section{Solutions to Functions}
\begin{enumerate}
	\item False, let \( C = \{0,1\} \) and \(g(x) = 0\)... (write a surjective \(f\) and sets)
	\item We show the contrapositive. Suppose \(f\) not injective. Then \(\exists a_1, a_2 \in A : f(a_1) = f(a_2) \in B \). Clearly \( g(f(a_1)) = g(f(a_2)) \) (as \(g\) is a well-defined function), so \(g\circ f\) is not injective.
	\item False, let \( A = \{0, 1\}, B = \{0,1,2\}, C=\{0,1\}, f=\mathrm{id}, g(0)=g(2)=0, g(1)=1 \).
	\item Suppose \(f, g\) surjective, i.e. \( \forall b \in B \exists a \in A : f(a) = b \) and \( \forall c \in C \exists b \in B : g(b) = c \).\\
	Fix \( c \in C\). Want: \( \exists a \in A : g(f(a)) = c \). We have \(b \in B\) s.t. \( g(b) = c \). Similarly \( \exists a \in A : f(a) = b \), which is what was required.
\end{enumerate}


\end{document}
