\documentclass[10pt,a4paper]{article}

\usepackage{fullpage}

\usepackage{amsmath}
\usepackage{amssymb}

\makeatletter
\def\footnoterule{\kern-3\p@
  \hrule \@width 2in \kern 2.6\p@} % the \hrule is .4pt high
\makeatother

\begin{document}

\title{PMT Additional Exercises (Week 3)}
\author{Nicholas Sim}
\date{October 19th, 2017}
\maketitle{}


\section{Discrete Structures: Relations}
% sup/inf
% infinite union/intersection again (?)
% archimedean axiom (ex: between any two reals lies a rational)
% Partitions
% Relations (this week)

\begin{enumerate}
    \item (*) For the relation \(x \sim y \Leftrightarrow | x - y | < 1 \), state whether it is reflexive, symmetric, or transitive, given:
    \begin{enumerate}
        \item \(x, y \in \mathbb{R} \)
        \item \(x, y \in \mathbb{Z} \)
    \end{enumerate}

    \item Consider a finite set \( S \), where \( |S| = n\).
    \begin{enumerate}
        \item How many relations are there on \( S \)? (on your tutorial sheet)
        \item How many of them are reflexive? (also on your tutorial sheet)
        \item Symmetric?
        \item Reflexive and symmetric?
        \item Transitive? (don't spend more than 5 minutes on this)
    \end{enumerate}

    \item Consider any integer \( n \). Show that if \( n^2 \) is divisible by 3, then so is \( n \). \\
    (Hint: Consider the equivalence classes \( \mathbb{Z} / R_3 \).)

    \item (**) Hence, prove that \( \sqrt{3} \) is irrational.
\end{enumerate}


\section{Logic: Translation Exercises}

% nothing carried forward

Translate the statements from English to Logic, defining atoms where needed. \emph{[format thanks to Jeremy Kong]}

\begin{enumerate}
    \item If you write no answers, (then) you will get zero marks.
    \item Tony will deduct 10,000 marks if I don't write a base case.
    \item (*) I will only get stuck in Huxley building when I stay after midnight.
    \item (Consider: I will only get stuck in Huxley building if I stay after 10 p.m.)
    \item Unless I submit the coursework on time, I can \emph{only} get a pass mark.
    \item My cactus won't grow unless I give it enough water.
    \item Three burgers are sufficient to make me full.
    \item You need a kettle to make tea.
    \item All men are mortal, and Socrates is a man. Therefore Socrates is mortal.
    \item (**) I know that the cat in the box is alive or dead.
    \item The parade will go on, even if it rains.
\end{enumerate}

\footnoterule
\emph{(*) Starred questions are harder, but good for you}


\newpage

\section{Solutions to Relations Exercises}
\begin{enumerate}
    \item
    \begin{enumerate}
        \item Reflexive and symmetric, but not transitive, since \( 1 \sim 1.5 \sim 2 \), but \( 1 \nsim 2 \)
        \item Equivalence relation. This is \( = \) on \( \mathbb{Z} \)
    \end{enumerate}

    \item Number of relations on a finite set. It may help to visualise a \( n \times n \) matrix.
    \begin{enumerate}
        \item Recall that a relation is a subset of \( S \times S \). So \( 2 ^ {n^2} \).
        \item All elements must be related to themselves. So we fix \( n \) elements out of \( S \times S \) (since we \emph{must} choose them, and get \( 2 ^ {n^2 - n}\).
        \item ``About half'', since \( x_i \sim x_j \Rightarrow x_j \sim x_i \). We must take consider all the cases where \( i = j \) separately. Thus: \( 2 ^ {(n^2 - n) / 2 + n}\).
        \item This is just the same as the previous part, except we \emph{must} choose elements \( (x, x) \). \(2 ^ {(n^2 - n) / 2}\)
        \item Trick question! There isn't a closed-form formula for this, but you can calculate it recursively. https://oeis.org/A006905
    \end{enumerate}

    \item Consider the elements defined by quotient set \( \mathbb{Z} / R_3 \).\footnote{Some other mathematicians write \( \mathbb{Z} / 3 \mathbb{Z} \), but Steffen's way is easier to type}
    There are 3 elements: \( [0], [1], [2] \). Note: \( [a] \) represents the set of integers \( x \) such that \( x \equiv a \mod 3 \). \\
    We will show the contrapositive.
    Recall that \( n \) divisible by 3 if and only if \( n \in [0] \). Suppose \( n \) is not divisible by 3, then \( n \in [1] \cup [2] \). In both cases, \( n^2 \in [1] \). (show why this is so!).Then \( n^2 \) is not divisible by 3.

    \item This sort of problem probably isn't going to appear in your exam, but the skills you need to understand it will certainly help! \\
    Suppose \( \sqrt{3} = \frac{p}{q} \), where \(p \in \mathbb{Z}, q \in \mathbb{N} \setminus \{0\} \) as in Example 2.16 (2). Recall that we can represent rational numbers in ``lowest terms'', so \(p, q\) have no common factors. By squaring, \( 3 q^2 = p^2 \), so we have \( p^2 \) is divisible by 3. Hence \(p\) is divisible by 3. \\
    Now let \( p = 3p' \), and \( q^2 = 3p'^2 \). Now \(q\) is divisible by 3. But \(p,q\) have no common factors, so this is a contradiction.
\end{enumerate}


\section{Solutions to Translation Exercises}
There may be equally correct alternative translations.

\begin{enumerate}
    \item \( \textrm{`write no answers'} \rightarrow \textrm{`zero marks'} \) (if P then Q)
    \item \( \textrm{`no base case'} \rightarrow \textrm{`lose 10,000 marks'} \) (Q if P)
    \item \( \textrm{`stuck in Huxley'} \rightarrow \textrm{`stayed after midnight'} \) (P only if (when) Q) \\
    Note: In this scenario, you would get stuck in Huxley precisely when you stayed after midnight. However, the example is only strong enough for a one-way implication! Hence the next exercise.
    \item \( \textrm{`stuck in Huxley'} \rightarrow \textrm{`stayed after 22.00'} \) (P only if Q)
    \item \( \neg \textrm{`submit on time'} \rightarrow \textrm{`only get pass mark'} \) (Q unless not P) \\
    In this context it is reasonable to use `submit late'. However, I would be more careful about changing the antecedent to `get more than a passing mark'.
    \item \( \textrm{`cactus grows'} \rightarrow \textrm{`enough water'} \) (not P unless Q, becoming: not Q implies not P, ``the contrapositive'')
    \item \( \textrm{`eat three burgers'} \rightarrow \textrm{`full'} \) (P is sufficient for Q) \\
    Maybe two burgers are sufficient! Who knows?
    \item \( \textrm{`make tea'} \rightarrow \textrm{`have kettle'} \) (Q is necessary for P)
    \item \( \textrm{`all men are mortal'} \land \textrm{`Socrates is a man'} \rightarrow \textrm{`Socrates is mortal'} \) \\
    Rather dissatisfyingly, we can't do better with propositional logic. Later, with predicate logic, we'll learn to `split the atom'.
    \item \( \textrm{`I know that the cat in the box is alive or dead'}\) \\
    We can't reduce this! Consider: \( \textrm{`I know the cat is alive'} \lor \textrm{`I know the cat is dead'}\).
    If I am Schr{\"o}dinger, the two terms might be false (I don't know which is true), but I do know that cats must be either alive or dead.
    \item \( \textrm{`the parade will go on'}\) \\
    Here, it doesn't matter whether it rains or not.
\end{enumerate}

\end{document}