\documentclass[10pt,a4paper]{article}
    
\usepackage{fullpage}

\usepackage{amsmath}
\usepackage{amssymb}

\begin{document}

\title{PMT Additional Exercises (Week 7)}
\author{Nicholas Sim}
\date{November 16th, 2017}
\maketitle{}
% We skipped Week 6 to do some ND and exam questions.

% Remember to do some proofs with them.
% Predict that the DS assignment will be poorly attempted. 
% These proofs should be sufficient.


\section{Logic: First-order translations}
\subsection{Easy translations}
Defining appropriate constants, translate the following. You may use the following relations:

\begin{itemize}
    \item person(\(x\)) means \(x\) is a person
    %\item happy(\(x\)) means \(x\) is happy
    \item hungry(\(x\)) means \(x\) is hungry
    \item loves(\(x, y\)) means \(x\) loves \(y\)
\end{itemize}

We define an ordering.
\begin{enumerate}
    \item All love is unreciprocated. \emph{(anti-symmetry)}
    \item If a person's beloved loves another, then so does the original person. \emph{(transitivity)}
    \item Nobody loves themselves. \emph{(anti-reflexivity)}
    \item Between any two people, one must love the other. \emph{(totality)}
    \item Everybody loves somebody. \emph{(minimum element)} \\

    Some mathematical translations (you may assume the ordering above holds)
    \item There is a hungry person who is loved by all other hungry people.

    \item There are (at least) three hungry people.

    \item There are at most two hungry people.

    \item There are infinitely many hungry people.
\end{enumerate}

You might notice that unary relations can define sets (like the set of hungry people).

% Construct groups with relations instead of functions.

It's hard to discuss mathematics without functions.
Once we have these we can construct algebraic structures, such as groups, 
with predicate logic.\footnote{Or we can (painfully!) use relations to stand in for functions, as you'll see next term.}

% Construct groups next time
\subsection{Graphs (also easy)}
Suppose L is a language with equality and a 2-ary relation symbol \(E\).

Define a first-order L-structure \(G\), 
calling the elements in its domain \emph{vertices}, 
using constant symbols (of L) \(v_1, v_2, ...\).
Write a formula such that \texttt{E} has the following properties in \(G\):

\begin{enumerate}
    \item \texttt{E} is irreflexive
    \item \texttt{E} is symmetric
\end{enumerate}

Call \(G\) an undirected graph. Translate the following:

\begin{enumerate}
    \item \(G\) is a complete graph
    \item Any two vertices of \(G\) share exactly 1 common neighbour
\end{enumerate}

\emph{(Erd\"os-Renyi friendship theorem:)} In a finite graph, 
if any two vertices share precisely one common neighbour, 
then the graph is a \emph{friendship graph} (or Dutch windmill graph).

Notice that we normally define the set of edges as a subset of \(V \times V\). 
Like how a unary relation defines a subset of the domain (\(V\) here), 
a binary relation defines a subset of \(V \times V\).

Why can't we express the following?
\begin{enumerate}
    \item There is a path between \(v_1\) and \(v_2\)
    \item \(G\) is a connected graph
\end{enumerate}

\newpage

\newcommand{\loves}[1]{\textrm{loves}(#1)}
\newcommand{\person}[1]{\textrm{person}(#1)}
\newcommand{\hungry}[1]{\textrm{hungry}(#1)}

\section{Solutions to translations}
\subsection{Easy translations}
\begin{enumerate}
    \item \( (\forall x) (\forall y) \left( \loves{x, y} \rightarrow \neg \loves{y, x} \right) \).
    \item Note that I only require the original lover to be a person. \\
    \( (\forall x) (\forall y) (\forall z) \left( 
        \person{x} \land \loves{x, y} \land \loves{y, z} \rightarrow \loves{x, z} \right) \)
    \item \( (\forall x) \left( \person{x} \rightarrow \neg \loves{x, x} \right) \)
    \item \( (\forall x) (\forall y) \left(
        \person{x} \land \person{y} \rightarrow \loves{x, y} \lor \loves{y, x} \right) \)
    \item \( (\exists x) (\forall y) \loves{y, x} \) \\

    Important: for brevity below, I have declined to state that \(x, y, z\) may be people.
    \item Realise that \emph{hungriness} defines a subset of \emph{people}.
    In fact, so does any unary relation.\\
    \( (\exists x) \left( \hungry{x} \land (\forall y) \left( 
        y \ne x \land \hungry{y} \rightarrow \loves{y, x} \right) \right) \)
    Note: \(y \ne x\) is very important here.

    \item \( (\exists x) (\exists y) (\exists z) \left( x \ne y \land y \ne z \land 
        x \ne z \land \hungry{x} \land \hungry{y} \land \hungry{z} \right) \)

    \item Negate the last statement.

    \item Essentially, we are saying there is no finite number of hungry people.
    However, it would quickly get boring writing a formula similar to the previous parts. \\
    \( \neg (\exists x) \left( \hungry{x} \land 
        (\forall y) \left( \hungry{y} \rightarrow \loves{x, y} \right) \right) \) (why?)

    Or use induction.
    \( (\forall x) \left( \hungry{x} \rightarrow (\exists y) \left(
        \hungry{y} \land \loves{y, x} \right) \right) \land 
        (\exists z) \hungry{z} \)

    Satisfy yourself that these are equivalent.
\end{enumerate}


\subsection{Graph translations}
\begin{enumerate}
    \item \( (\forall v_1) (\forall v_2) E(v_1, v_2) \)
    \item \( (\forall v_1) (\forall v_2) (\exists v_3) (E(v_1, v_3) \land E(v_3, v_2)) \)
\end{enumerate}

Why not:
\begin{enumerate}
    \item Essentially, we are talking about the transitive closure of \(E\).
    If we defined a new 2-ary relation \texttt{path}, it might look like this:
    \( path(u, v) : E(u, v) \lor (\exists t) (path(u, t) \land E(t, v)) \). \\
    Since we can't define add a new relation symbol to L, 
    you'll quickly see that there's a way to express that there is a path of length \(n\) between two vertices,
    there isn't (finite length) formula that can express that there is a path between two vertices.
    \item This follows from the previous section.
    The definition of a graph being connected is: pick any two vertices, then there is a path between them.
\end{enumerate}


\end{document}