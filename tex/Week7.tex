\documentclass[10pt,a4paper]{article}
    
\usepackage{fullpage}

\usepackage{amsmath}
\usepackage{amssymb}

\begin{document}

\title{PMT Additional Exercises (Week 7)}
\author{Nicholas Sim}
\date{November 16th, 2017}
\maketitle{}
% We skipped Week 6 to do some ND and exam questions.

% Remember to do some proofs with them.
% Predict that the DS assignment will be poorly attempted. 
% These proofs should be sufficient.


\section{Logic: First-order translations}
% I'm a little concerned about timing here. Delay this by a week?
Defining appropriate constants, translate the following. You may use the following relations:

\begin{itemize}
    \item person(\(x\)) means \(x\) is a person
    %\item happy(\(x\)) means \(x\) is happy
    \item hungry(\(x\)) means \(x\) is hungry
    \item loves(\(x, y\)) means \(x\) loves \(y\)
\end{itemize}

We define an ordering.
\begin{enumerate}
    \item All love is unreciprocated. \emph{(anti-symmetry)}
    \item If a person's beloved loves another, then so does the original person. \emph{(transitivity)}
    \item Nobody loves themselves. \emph{(anti-reflexivity)}
    \item Between any two people, one must love the other. \emph{(totality)}
    \item Everybody loves somebody. \emph{(minimum element)} \\

    Some mathematical translations (you may assume the ordering above holds)
    \item There is a hungry person who is loved by all other hungry people.

    \item There are (at least) three hungry people.

    \item There are at most two hungry people.

    \item There are infinitely many hungry people.
\end{enumerate}

You might notice that unary relations can define sets (like the set of hungry people).

% Construct groups with relations instead of functions.

It's hard to discuss mathematics without functions.
Once we have these we can construct algebraic structures, such as groups, with predicate logic.

% Construct groups next time

\newpage

\newcommand{\loves}[1]{\textrm{loves}(#1)}
\newcommand{\person}[1]{\textrm{person}(#1)}
\newcommand{\hungry}[1]{\textrm{hungry}(#1)}

\section{Solutions to translations}
\begin{enumerate}
    \item \( (\forall x) (\forall y) \left( \loves{x, y} \rightarrow \neg \loves{y, x} \right) \).
    \item Note that I only require the original lover to be a person. \\
    \( (\forall x) (\forall y) (\forall z) \left( 
        \person{x} \land \loves{x, y} \land \loves{y, z} \rightarrow \loves{x, z} \right) \)
    \item \( (\forall x) \left( \person{x} \rightarrow \neg \loves{x, x} \right) \)
    \item \( (\forall x) (\forall y) \left(
        \person{x} \land \person{y} \rightarrow \loves{x, y} \lor \loves{y, x} \right) \)
    \item \( (\exists x) (\forall y) \loves{y, x} \) \\

    Important: for brevity below, I have declined to state that \(x, y, z\) may be people.
    \item Realise that \emph{hungriness} defines a subset of \emph{people}.
    In fact, so does any unary relation.\\
    \( (\exists x) \left( \hungry{x} \land (\forall y) \left( 
        y \ne x \land \hungry{y} \rightarrow \loves{y, x} \right) \right) \)
    Note: \(y \ne x\) is very important here.

    \item \( (\exists x) (\exists y) (\exists z) \left( x \ne y \land y \ne z \land 
        x \ne z \land \hungry{x} \land \hungry{y} \land \hungry{z} \right) \)

    \item Negate the last statement.

    \item Essentially, we are saying there is no finite number of hungry people.
    However, it would quickly get boring writing a formula similar to the previous parts. \\
    \( \neg (\exists x) \left( \hungry{x} \land 
        (\forall y) \left( \hungry{y} \rightarrow \loves{x, y} \right) \right) \) (why?)

    Or use induction.
    \( (\forall x) \left( \hungry{x} \rightarrow (\exists y) \left(
        \hungry{y} \land \loves{y, x} \right) \right) \land 
        (\exists z) \hungry{z} \)

    Satisfy yourself that these are equivalent.
\end{enumerate}


\end{document}